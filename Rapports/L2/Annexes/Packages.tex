%% ====== Classe du document ============================================
\documentclass[10pt,a4paper]{article}
%\usepackage[left=1.5cm,right=1.5cm,top=2cm,bottom=2cm]{geometry}

%% ====== Francisation ==================================================
\usepackage[french]{babel}
\usepackage[T1]{fontenc}
\usepackage[utf8]{inputenc}
\usepackage{textcomp}

%% ====== Personnalisation ============================================
\usepackage{fancyhdr}
	\lhead{}
	\chead{Projet MDMA - Rapport L2}
	\rhead{M1 IF 2012-2013}
	\renewcommand{\headrulewidth}{0.3pt}
	\renewcommand{\footrulewidth}{0.3pt}
	\lfoot {}
	\cfoot {- \thepage -}
	\rfoot {}
	\pagestyle{fancy}
\title{Projet MDMA - Rapport L2}
\author{}
\date{}

\usepackage{hyperref}

%% ====== Packages pour le texte ========================================
%\usepackage{soul}
%\usepackage[normalem]{ulem}
%\usepackage{fancybox}
%\usepackage{moreverb}
%\usepackage[table]{xcolor}
%% ====== Packages pour les dessins =====================================
\usepackage{float}
\usepackage{graphicx}
%\usepackage{multicol}
%\usepackage{multirow}
%\usepackage{tikz}
%\usepackage{lmodern}
\usepackage{pict2e}

%% ====== Packages pour les maths =======================================
%\usepackage{amsmath}
%\usepackage{amssymb}
%\usepackage{mathrsfs}
%\usepackage{bussproofs}
%\usepackage[ruled,vlined,french]{algorithm2e}

%%% francisation des algorithmes

%\usepackage[squaren,Gray]{SIunits}
%% ====== Reglages generaux =============================================

\usepackage{titlesec}
	\titleformat{\section}[frame]
	{\normalfont}
	{\filright
	\footnotesize
	\enspace Partie \thesection\enspace}
	{6pt}
	{\bfseries\filcenter}
	
	\titleformat{\subsection}[frame]
	{\normalfont}
	{\filright
	\footnotesize
	\enspace \thesubsection\enspace}
	{6pt}
	{\filcenter}
%	{\titlerule
%	\vspace{.8ex}%
%	\normalfont\itshape}
%	{\thesubsection.}{.5em}{}

	\titleformat{\subsubsection}
	{\titlerule
	\vspace{.8ex}%
	\normalfont\itshape}
	{}{.5em}{}

\titleformat{\chapter}[display]
	{\normalfont\bfseries\filcenter}
	{}
	{1ex}
	{\titlerule[2pt]
	\vspace{2ex}%
	\LARGE}
	[\vspace{1ex}%
	{\titlerule[2pt]}]
	
\parindent=10pt

%\usepackage{makeidx}
%\makeindex
%\newcommand\vect{\overrightarrow}

%\numberwithin{equation}{subsection}


\graphicspath{{img/}}
